\usepackage{xcolor}
\usepackage{afterpage}
\usepackage{pifont,mdframed}
\usepackage[bottom]{footmisc}
\usepackage{algorithm}
\usepackage[noend]{algpseudocode}
\usepackage[colorlinks = true,linkcolor = black,urlcolor  = blue,citecolor = black,anchorcolor = black]{hyperref}


\newcommand{\inputfile}{\texttt{input.txt}}
\newcommand{\outputfile}{\texttt{output.txt}}

\newenvironment{warning}
  {\par\begin{mdframed}[linewidth=2pt,linecolor=gray]%
    \begin{list}{}{\leftmargin=1cm
                   \labelwidth=\leftmargin}\item[\Large\ding{43}]}
  {\end{list}\end{mdframed}\par}

% % % % % % % % % % % % % % % % % % % % % % % % % % % % % % % % % % % % % % % % % % %
% % % % % % % % % % % % % % % % % % % % % % % % % % % % % % % % % % % % % % % % % % %

	{
	\vspace{-.95cm}\hfill\fbox{Difficoltà: 2}
	}
	\vspace{.5cm}

	Gabriele ha un nuovo rompicapo preferito, chiamato ``Rispetta i versi''. Si tratta di un solitario giocato su una griglia formata da $N$ caselle separate da un simbolo di disuguaglianza; in figura è mostrato un esempio con $N = 6$.
	
	\vspace{-1mm}
	\begin{figure}[h]
	\centering\colorbox{black!10!white}{
		\includegraphics[scale=.8]{extra_disuguaglianze/asy_disuguaglianze/fig0.pdf}
	}
	\end{figure}

	L'obiettivo del gioco è quello di riempire le celle vuote con tutti i numeri da $1$ a $N$ (ogni numero deve comparire esattamente una volta), in modo da rispettare le disuguaglianze tra caselle adiacenti. Per la griglia della figura, una delle possibili soluzioni al rompicapo è la seguente:
	
	\vspace{-1mm}
	\begin{figure}[h]
	\centering\colorbox{black!10!white}{
		\includegraphics[scale=.8]{extra_disuguaglianze/asy_disuguaglianze/fig1.pdf}
	}
	\end{figure}
	\vspace{-1mm}
	
% % % % % % % % % % % % % % % % % % % % % % % % % % % % % % % % % % % % % % % % % % %
% % % % % % % % % % % % % % % % % % % % % % % % % % % % % % % % % % % % % % % % % % %

	\InputFile
	Il file \inputfile{} contiene due righe di testo. Sulla prima è presente l'intero $N$, il numero di caselle del gioco. Sulla seconda è presente una stringa di $N-1$ caratteri, ognuno dei quali può essere solo \verb|<| o \verb|>|, che descrive i vincoli tra le caselle, da sinistra a destra.


	\OutputFile
	Il file \outputfile{} contiene su una sola riga una qualunque permutazione dei numeri da $1$ a $N$ - separati tra loro da uno spazio - che risolve il rompicapo. I numeri corrispondono ai valori scritti nelle caselle, leggendo da sinistra verso destra.
	
% % % % % % % % % % % % % % % % % % % % % % % % % % % % % % % % % % % % % % % % % % %
% % % % % % % % % % % % % % % % % % % % % % % % % % % % % % % % % % % % % % % % % % %


\Constraints

\begin{itemize}[nolistsep, itemsep=2mm]
	\item $2 \le N \le 100\,000$.
	\item Nel 30\% dei casi, il valore di $N$ non supera 10.
	\item Nel 60\% dei casi, il valore di $N$ non supera 20.
	\item Si garantisce l'esistenza di almeno una soluzione per ciascuno dei casi di test utilizzati nella verifica del funzionamento del programma.
\end{itemize}

% % % % % % % % % % % % % % % % % % % % % % % % % % % % % % % % % % % % % % % % % % %
% % % % % % % % % % % % % % % % % % % % % % % % % % % % % % % % % % % % % % % % % % %

\Examples

\begin{example}
\exmp{\verbatiminput{disuguaglianze0.txt}}{
2 5 1 3 6 4
}%
\exmp{
5
>{}>{}<{}<}{
5 3 1 2 4
}%
\exmp{
8
>{}>{}<{}>{}>{}<{}>
}{%
6 5 4 7 3 2 8 1
}%
\end{example}

% % % % % % % % % % % % % % % % % % % % % % % % % % % % % % % % % % % % % % % % % % %
% % % % % % % % % % % % % % % % % % % % % % % % % % % % % % % % % % % % % % % % % % %

\newpage
\begin{solution}
	Consideriamo un arco $(u, v)$ e cerchiamo il più breve ciclo di cui esso fa parte.
I cicli semplici contenenti l'arco $(u, v)$ nel grafo della città corrispondono evidentemente ai cammini semplici che congiungono il nodo $u$ al nodo $v$ in un grafo a cui è stato eliminato l'arco $(u, v)$.

È facile trovare la lunghezza del cammino minimo tra $u$ e $v$ nel grafo ``ridotto'' utilizzando l'\href{https://it.wikipedia.org/wiki/Algoritmo\_di\_Dijkstra}{algoritmo di Dijkstra}\footnote{\url{http://it.wikipedia.org/wiki/Algoritmo_di_Dijkstra}}. La lunghezza del ciclo è quindi la distanza trovata da Dijkstra, sommata alla lunghezza dell'arco $(u, v)$.

A questo punto risolvere il problema è semplice: per ogni arco troviamo il ciclo più corto che lo contiene e la risposta sarà il minimo delle lunghezze di questi cicli.

La complessità computazionale di questo algoritmo è $O(M(M+N\log N))$, dato che esegue $M$ volte l'algoritmo di Dijkstra che ha complessità computazionale $O(M+N\log N)$.

Approfondimento: è possibile modificare la soluzione precedente, basandosi sempre sull'algoritmo di Dijkstra, e ottenere una soluzione $O(NM + N^2\log N)$. Si ottiene comunque un notevole miglioramento in prestazioni (per quanto non in complessità computazionale) facendo in modo che l'algoritmo di Dijkstra non esplori cammini più lunghi del ciclo minimo già trovato.

\createsection{\Codice}{Esempio di codice \texttt{C++11}}
\Codice

\colorbox{white}{\makebox[.99\textwidth][l]{
    \includegraphics[trim=0 595 0 0, clip, scale=.73]{extra_footing/codice_footing.pdf}
}}
\colorbox{white}{\makebox[.99\textwidth][l]{
    \includegraphics[trim=0 0 0 458, clip, scale=.73]{extra_footing/codice_footing.pdf}
}}

\end{solution}
