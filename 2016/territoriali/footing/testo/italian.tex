\usepackage{xcolor}
\usepackage{afterpage}
\usepackage{pifont,mdframed}
\usepackage[bottom]{footmisc}
\usepackage{wrapfig}
\usepackage[colorlinks = true,linkcolor = black,urlcolor  = blue,citecolor = black,anchorcolor = black]{hyperref}


\newcommand{\inputfile}{\texttt{input.txt}}
\newcommand{\outputfile}{\texttt{output.txt}}

\newenvironment{warning}
  {\par\begin{mdframed}[linewidth=2pt,linecolor=gray]%
    \begin{list}{}{\leftmargin=1cm
                   \labelwidth=\leftmargin}\item[\Large\ding{43}]}
  {\end{list}\end{mdframed}\par}

% % % % % % % % % % % % % % % % % % % % % % % % % % % % % % % % % % % % % % % % % % %
% % % % % % % % % % % % % % % % % % % % % % % % % % % % % % % % % % % % % % % % % % %

	{
	\vspace{-.95cm}\hfill\fbox{Difficoltà: 2}
	}
	\vspace{.5cm}

	William sta pensando di trasferirsi in una nuova città e vuole selezionare, tra le varie possibilità, quella che si concilia meglio con la sua routine mattutina. Infatti, William è abituato a fare una corsetta attorno al proprio isolato tutte le mattine, e teme che traslocando debba rinunciare a questo hobby, qualora l'isolato in cui verrebbe a trovarsi fosse troppo grande.

	\begin{wrapfigure}{r}{0.3\textwidth}
	  \begin{center}
	    \includegraphics[width=0.25\textwidth]{extra_footing/asy_footing/fig1.pdf}
	  \end{center}
	  \caption{\emph{La mappa della città descritta nel primo input di esempio.}}
	\end{wrapfigure}

	La mappa della città si può rappresentare come un insieme di strade e di incroci tra queste. A ogni incrocio c'è una casa e le strade possono essere percorse in entrambi i sensi. Le case sono numerate da $1$ a $N$.
	Per evitare di annoiarsi, William non ha intenzione di fare corsette che passino due volte davanti alla stessa casa, ad eccezione della sua (infatti la corsetta deve necessariamente cominciare e terminare nella stessa casa). Questo tipo di percorso prende il nome di \emph{ciclo semplice}.

	Nonostante i buoni propositi, William è molto pigro; per questo motivo ha intenzione di rendere la sua corsetta mattutina il più breve possibile: aiutalo  scrivendo un programma che prenda in input la mappa di una città e determini la lunghezza del \emph{ciclo semplice} più corto. Con questa informazione, William potrà decidere se trasferirsi nella nuova città, ovviamente solo se riuscirà poi ad andare ad abitare in una delle case  che appartengono a questo percorso.


	Si prenda ad esempio la mappa della città in Figura 1 (dove il numero a fianco di ogni strada indica la lunghezza della strada), alcuni dei suoi cicli semplici sono i seguenti:

	\begin{figure}[h!]
	  \centering
	  \includegraphics[width=2in]{extra_footing/asy_footing/fig2.pdf}\hfill
	  \includegraphics[width=2in]{extra_footing/asy_footing/fig3.pdf}\hfill
	  \includegraphics[width=2in]{extra_footing/asy_footing/fig4.pdf}
	\end{figure}


	Come si può vedere, i primi due cicli evidenziati hanno una lunghezza totale pari a $9$, il terzo invece ha una lunghezza pari a $8$ ed è quindi il percorso ottimale per la corsetta mattutina di William: adesso William sa quali sono le case coinvolte nel percorso più breve, e tra quelle potrà cercare la nuova casa in cui andare ad abitare.

% % % % % % % % % % % % % % % % % % % % % % % % % % % % % % % % % % % % % % % % % % %
% % % % % % % % % % % % % % % % % % % % % % % % % % % % % % % % % % % % % % % % % % %

	\InputFile
	Il file \inputfile{} contiene $M+1$ righe di testo. Sulla prima sono presenti due interi separati da spazio: $N$ e $M$, rispettivamente il numero di case ed il numero di tratti di strada presenti nella città. Dalla riga $2$ fino alla $M+1$ troviamo la descrizione degli $M$ tratti di strada. Ciascuna di queste righe contiene tre interi separati da spazio: $u$, $v$ e $w$, dove $u$ e $v$ sono due case (quindi sono degli indici compresi tra $1$ ed $N$) e $w$ è la lunghezza del tratto di strada che le collega.


	\OutputFile
	Il file \outputfile{} contiene un singolo intero: la lunghezza del \emph{ciclo semplice} più corto presente nella città in input.

% % % % % % % % % % % % % % % % % % % % % % % % % % % % % % % % % % % % % % % % % % %
% % % % % % % % % % % % % % % % % % % % % % % % % % % % % % % % % % % % % % % % % % %


\Constraints

\begin{itemize}[nolistsep, itemsep=2mm]
	\item $3 \le N \le 1000$.
	\item $3 \le M \le 10\,000$.
	\item $0 < w \le 10\,000$, dove $w$ è la lunghezza di un tratto di strada.
	\item È garantito che nella città esiste sempre almeno un ciclo semplice.
%	\item È garantito che una coppia di case adiacenti è collegata da \emph{un solo} tratto di strada.  % TODO
	\item Nel 40\% dei casi di prova tutte le strade hanno lunghezza unitaria.
	\item È garantito che una coppia di case adiacenti è collegata da \emph{un solo} tratto di strada.
	\item Una strada non collega mai una casa a se stessa.
\end{itemize}

% % % % % % % % % % % % % % % % % % % % % % % % % % % % % % % % % % % % % % % % % % %
% % % % % % % % % % % % % % % % % % % % % % % % % % % % % % % % % % % % % % % % % % %

\Examples

\begin{example}
\exmp{\verbatiminput{footing0.txt}}{
8
}%
%TODO: aggiungere altri esempi
\end{example}

\Notes
\begin{itemize}[nolistsep, itemsep=2mm]
	\item Per chi usa Pascal: è richiesto che si utilizzi sempre il tipo di dato \verb|longint| al posto di \verb|integer|.
	\item Un programma che stampa lo stesso output indipendentemente dal file di input non totalizza alcun punteggio.
\end{itemize}

% % % % % % % % % % % % % % % % % % % % % % % % % % % % % % % % % % % % % % % % % % %
% % % % % % % % % % % % % % % % % % % % % % % % % % % % % % % % % % % % % % % % % % %

\newpage
\begin{solution}
    Consideriamo un arco $(u, v)$ e cerchiamo il più breve ciclo di cui esso fa parte.
I cicli semplici contenenti l'arco $(u, v)$ nel grafo della città corrispondono evidentemente ai cammini semplici che congiungono il nodo $u$ al nodo $v$ in un grafo a cui è stato eliminato l'arco $(u, v)$.

È facile trovare la lunghezza del cammino minimo tra $u$ e $v$ nel grafo ``ridotto'' utilizzando l'\href{https://it.wikipedia.org/wiki/Algoritmo\_di\_Dijkstra}{algoritmo di Dijkstra}\footnote{\url{http://it.wikipedia.org/wiki/Algoritmo_di_Dijkstra}}. La lunghezza del ciclo è quindi la distanza trovata da Dijkstra, sommata alla lunghezza dell'arco $(u, v)$.

A questo punto risolvere il problema è semplice: per ogni arco troviamo il ciclo più corto che lo contiene e la risposta sarà il minimo delle lunghezze di questi cicli.

La complessità computazionale di questo algoritmo è $O(M(M+N\log N))$, dato che esegue $M$ volte l'algoritmo di Dijkstra che ha complessità computazionale $O(M+N\log N)$.

Approfondimento: è possibile modificare la soluzione precedente, basandosi sempre sull'algoritmo di Dijkstra, e ottenere una soluzione $O(NM + N^2\log N)$. Si ottiene comunque un notevole miglioramento in prestazioni (per quanto non in complessità computazionale) facendo in modo che l'algoritmo di Dijkstra non esplori cammini più lunghi del ciclo minimo già trovato.

\createsection{\Codice}{Esempio di codice \texttt{C++11}}
\Codice

\colorbox{white}{\makebox[.99\textwidth][l]{
    \includegraphics[trim=0 595 0 0, clip, scale=.73]{extra_footing/codice_footing.pdf}
}}
\colorbox{white}{\makebox[.99\textwidth][l]{
    \includegraphics[trim=0 0 0 458, clip, scale=.73]{extra_footing/codice_footing.pdf}
}}

\end{solution}
