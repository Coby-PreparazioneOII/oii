\usepackage{xcolor}
\usepackage{afterpage}
\usepackage{pifont,mdframed}
\usepackage[bottom,symbol]{footmisc}

\createsection{\Input}{Dati di input}
\createsection{\Output}{Dati di output}

\newcommand{\inputfile}{\texttt{input.txt}}
\newcommand{\outputfile}{\texttt{output.txt}}

% % % % % % % % % % % % % % % % % % % % % % % % % % % % % % % % % % % % % % % % % % %
% % % % % % % % % % % % % % % % % % % % % % % % % % % % % % % % % % % % % % % % % % %

Nel film Totò Le Mokò, Totò ha un modo peculiare di dividere le gemme rubate con un suo complice:

\begin{itemize}
  \item inizia dicendo ``una a me'' (e se ne prende una),
  \item poi dice ``una a te'' (e ne dà una al complice),
  \item poi dice ``due a me'' (e se ne prende \textbf{due}),
  \item poi dice ``due a te'' (ma ne dà \textbf{solo una} al complice),
  \item poi dice ``tre a me'' (e se ne prende \textbf{tre}),
  \item poi dice ``tre a te'' (ma ne dà \textbf{solo una} al complice),
  \item e così via\dots
\end{itemize}

Totò inizia \emph{sempre} la spartizione prendendo una gemma per sé. Per esempio, se ci sono $11$ gemme da spartire, Totò ne prende $8$ e il suo complice $3$: la prima volta ne prendono una per uno, poi Totò due e il complice una, poi Totò tre e il complice una, infine Totò prende le due rimanenti (e nessuna gemma per il complice).

La prima volta che Totò ha fatto questa spartizione il complice ha protestato, ma Totò gli ha mollato un ceffone e gli ha preso le gemme che gli aveva dato; da allora nessuno osa contraddire Totò Le Mokò in una spartizione.

Le regole della spartizione sono le stesse anche se ci sono più complici con cui dividere il bottino: ad esempio, se ci sono $16$ gemme da dividere in quattro (Totò e tre complici), Totò ne prende $7$ e i tre complici ne prendono $3$ ciascuno: la prima volta ne prendono una per uno, poi Totò due e i complici una ciascuno, poi Totò tre e i complici una ciascuno, infine Totò prende la gemma rimanente.

Quando ci sono tante gemme Totò ha paura di sbagliarsi nella spartizione, quindi il vostro compito è quello di scrivere un programma che, ricevuti in ingresso il numero di gemme e il numero di persone (compreso Totò) tra cui spartirle, calcoli il numero di gemme che rimangono a Totò.

\Input
Il file \inputfile{} è composto da una riga contenente $G$ e $P$, due interi positivi rappresentanti rispettivamente il numero di gemme e il numero di persone (compreso Totò) tra cui spartirle.

\Output
Il file \outputfile{} è composto da una sola riga contenente un
intero positivo $T$: il numero di gemme che rimane a Totò dopo la spartizione.

\pagebreak
\Constraints
\begin{itemize}[nolistsep, itemsep=2mm]
  \item $10 \le G \le 1000$.
  \item $2 \le P \le 10$.
  \item Nel $50\%$ dei casi di input $P=2$.
\end{itemize}

\Examples
\begin{example}
\exmpfile{spartizione.input0.txt}{spartizione.output0.txt}%
\exmpfile{spartizione.input1.txt}{spartizione.output1.txt}%
\end{example}
