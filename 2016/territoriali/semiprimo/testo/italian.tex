\usepackage{xcolor}
\usepackage{afterpage}
\usepackage{pifont,mdframed}
\usepackage[bottom]{footmisc}
\usepackage{amsthm}

\newcommand{\inputfile}{\texttt{input.txt}}
\newcommand{\outputfile}{\texttt{output.txt}}

\newenvironment{warning}
  {\par\begin{mdframed}[linewidth=2pt,linecolor=gray]%
    \begin{list}{}{\leftmargin=1cm
                   \labelwidth=\leftmargin}\item[\Large\ding{43}]}
  {\end{list}\end{mdframed}\par}

% % % % % % % % % % % % % % % % % % % % % % % % % % % % % % % % % % % % % % % % % % %
% % % % % % % % % % % % % % % % % % % % % % % % % % % % % % % % % % % % % % % % % % %

	{
	\vspace{-.95cm}\hfill\fbox{Difficoltà: 1}
	}
	\vspace{.5cm}

	Gemma ha appena imparato che cos'è un numero semiprimo, e presa dall'euforia non riesce a smettere di parlarne. In particolare, \textbf{un numero semiprimo è un intero $\ge 2$ che si fattorizza come prodotto di due numeri primi} (non necessariamente distinti).

\begin{warning}
	I numeri primi sono tutti quegli interi $\ge 2$ divisibili solo per se stessi e per $1$. 
\end{warning}

	Sono quindi esempi di numeri semiprimi i numeri:
	\begin{itemize}[nolistsep, itemsep=2mm]
		\item $15$, prodotto di $3$ e $5$.
		\item $169$, prodotto di $13$ e $13$.
	\end{itemize}

	Aiuta Gemma a scrivere un programma che verifichi se un numero $N$ è semiprimo oppure no!

% % % % % % % % % % % % % % % % % % % % % % % % % % % % % % % % % % % % % % % % % % %
% % % % % % % % % % % % % % % % % % % % % % % % % % % % % % % % % % % % % % % % % % %

	\InputFile
	Il file \inputfile{} contiene l'unico intero $N$, di cui Gemma vuole verificare la semiprimalità.


	\OutputFile
	Il file \outputfile{} contiene:
	\begin{itemize}[nolistsep, itemsep=2mm]
		\item I due primi che fattorizzano $N$, stampati su un'unica riga, in ordine non-decrescente, se $N$ è semiprimo.
		\item L'unico intero $-1$ se $N$ \textbf{non} è semiprimo.
	\end{itemize}
	
% % % % % % % % % % % % % % % % % % % % % % % % % % % % % % % % % % % % % % % % % % %
% % % % % % % % % % % % % % % % % % % % % % % % % % % % % % % % % % % % % % % % % % %


\Constraints

\begin{itemize}[nolistsep, itemsep=2mm]
	\item $2 \le N \le 1\,000\,000$.
\end{itemize}

% % % % % % % % % % % % % % % % % % % % % % % % % % % % % % % % % % % % % % % % % % %
% % % % % % % % % % % % % % % % % % % % % % % % % % % % % % % % % % % % % % % % % % %

\Examples

\begin{example}
\exmp{\verbatiminput{semiprimo0.txt}}{
31 31
}%
\exmp{
884053}{
101 8753
}%
\exmp{
16
}{%
-1
}%
\end{example}

% % % % % % % % % % % % % % % % % % % % % % % % % % % % % % % % % % % % % % % % % % %
% % % % % % % % % % % % % % % % % % % % % % % % % % % % % % % % % % % % % % % % % % %

\newpage
\begin{solution}
    Consideriamo un arco $(u, v)$ e cerchiamo il più breve ciclo di cui esso fa parte.
I cicli semplici contenenti l'arco $(u, v)$ nel grafo della città corrispondono evidentemente ai cammini semplici che congiungono il nodo $u$ al nodo $v$ in un grafo a cui è stato eliminato l'arco $(u, v)$.

È facile trovare la lunghezza del cammino minimo tra $u$ e $v$ nel grafo ``ridotto'' utilizzando l'\href{https://it.wikipedia.org/wiki/Algoritmo\_di\_Dijkstra}{algoritmo di Dijkstra}\footnote{\url{http://it.wikipedia.org/wiki/Algoritmo_di_Dijkstra}}. La lunghezza del ciclo è quindi la distanza trovata da Dijkstra, sommata alla lunghezza dell'arco $(u, v)$.

A questo punto risolvere il problema è semplice: per ogni arco troviamo il ciclo più corto che lo contiene e la risposta sarà il minimo delle lunghezze di questi cicli.

La complessità computazionale di questo algoritmo è $O(M(M+N\log N))$, dato che esegue $M$ volte l'algoritmo di Dijkstra che ha complessità computazionale $O(M+N\log N)$.

Approfondimento: è possibile modificare la soluzione precedente, basandosi sempre sull'algoritmo di Dijkstra, e ottenere una soluzione $O(NM + N^2\log N)$. Si ottiene comunque un notevole miglioramento in prestazioni (per quanto non in complessità computazionale) facendo in modo che l'algoritmo di Dijkstra non esplori cammini più lunghi del ciclo minimo già trovato.

\createsection{\Codice}{Esempio di codice \texttt{C++11}}
\Codice

\colorbox{white}{\makebox[.99\textwidth][l]{
    \includegraphics[trim=0 595 0 0, clip, scale=.73]{extra_footing/codice_footing.pdf}
}}
\colorbox{white}{\makebox[.99\textwidth][l]{
    \includegraphics[trim=0 0 0 458, clip, scale=.73]{extra_footing/codice_footing.pdf}
}}

\end{solution}
